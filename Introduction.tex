\section{Introduction}

Acoustic source localization (ASL) is an important signal processing task, and has a wide range of application scenarios, such as speaker-location-aware audio capturing in videoconferencing \cite{guo2011localising}, shooter localization in a battle field \cite{sallai2011acoustic}, and wild biological acoustic studies \cite{allen2008voxnet}. 
The traditional centralized microphone array-based solution to ASL exploited multiple synchronized microphones to simultaneously acquire multiple signals, 
which had some limitations with regard to the distances between the microphones, and sensing range for the large-scale applications.
Wireless acoustic sensor networks (WASNs) can overcome these limitations. 
A WASN consists of a set of wireless microphone nodes that are spatially distributed over the environment, usually in an ad-hoc fashion. 
Due to the wireless communication capabilities, the array-size limitations disappear and the microphone nodes can physically cover a much larger area. 

% Wang, \emph{et al.} \cite{wang2003acoustic} described a system having static cluster architecture, the system experienced a problem in that the accuracy decreased when an acoustic source occurred between the clusters.
% Chen, \emph{et al.} \cite{chen2004dynamic} showed that nodes in the system did not need to recognize their cluster head, reducing the constraints on deployment of the localization system.
% Hu, et al. \cite{hu2009design} design the system based on 2-tier architecture, which experienced cost and deployment problems especially in the very large target area.
% Rabbat, \emph{et al.} \cite{rabbat2005robust} proposed a decentralized algorithm based on the distributed ML estimation technique using token ring architecture.
% Kim, \emph{et al.} \cite{kim2009locating} proposed to identify the node closest to the acoustic source, based on TOA comparisons between all nodes, thus incurring high communication cost and requiring global synchronization between all sensor nodes.
% Lightning is a method proposed in \cite{wang2008lightning} to identify the sensor closest to the acoustic source, also based on expensive broadcasting/flooding.
Acoustic source localization problem in sensor networks has been widely studied in the literature. 
There is a trade-off between the accuracy of localization and the computational complexity for existing different solutions.
Most of localization systems are based on range information, such as distance measurements and angle measurements among sensor nodes. 
Range-based localization methods can achieve good localization performance, however, generally are sensitive to the mesurement errors\cite{zhong2009tracking}. 
%\cite{yang2013freeloc} \cite{shu2015toc}
%The requirement of low cost and power prohibits many range-based methods for acoustic source localization, especially for large-scale deployments. 
Yedavalli, \emph{et al.}~\cite{yedavalli2008sequence} proposed a range-free Sequence-Based Localization (SBL) method in wireless sensor networks. 
The heart of SBL is the division of a 2D localization space into distinct regions by the perpendicular bisectors of lines joining pairs of anchor nodes.
Each distinct region can be uniquely identified by a node sequence that represents the distance ranks of the acoustic source to that region. 
Based on the rank of measurements between the acoustic source and the sensor nodes, the location of acoustic source can be estimated by searching through the node sequence table.
%Zhong, \emph{et al.} \cite{zhong2009tracking} converted the original tracking problem to the problem of finding the shortest path in a graph, which is equivalent to optimal matching of the node sequences. 
%Zhong, \emph{et al.} \cite{zhong2011rsd} introduced a proximity metric called RSD to capture the distance relationships among 1-hop neighboring nodes in a range-free manner. With little overhead, RSD can be conveniently applied for connectivity-based localization solutions to achieve better accuracy.

In this paper, we present a robust Half Plane Intersection to Sequence-Based Localization (HPI-SBL) by deeply mining the information embedded in the node sequence. 
As a range-free scheme, our design applies node sequences instead of direct time of arrival (TOA) as the measurement information, and has the following two major advantages: 
(i) node sequences are more robust to the measurement noise; 
(ii) node sequences significantly alleviate the accuracy requirement of time synchronization in sensor networks. 
Compared with earlier works on sequence-based localization in sensor
networks (e.g. SBL~\cite{yedavalli2008sequence}), the primary contribution
of our work is providing a robust approach to solve the sequence-based localization problem with the uncertainty of node position errors and measurement errors. 
The proposed HPI-SBL system formulates the localization as the probabilistic half-plane intersection problem. 
The proposed design is evaluated with both test-bed experiments and extensive simulations. 
Evaluation results show that the proposed HPI-SBL system can provide improved localization robustness.


The rest of the article is organized as follows. Section \uppercase\expandafter{\romannumeral 2} presents the localization system model.
Then, our methods are introduced in section \uppercase\expandafter{\romannumeral 3}.
Section \uppercase\expandafter{\romannumeral 4} presents simulation results . 
Section \uppercase\expandafter{\romannumeral 5} concludes the whole article.

% The remainder of the article is organized as follows. 
% Section \uppercase\expandafter{\romannumeral 2} presents an overview of the  localization system.
% Then, the basic HPI-SBL and advanced HPI-SBL is introduced in section \uppercase\expandafter{\romannumeral 3}.
% %Section \uppercase\expandafter{\romannumeral 4} discusses several practical issues.
% Section \uppercase\expandafter{\romannumeral 5} presents simulation results and an empirical evaluation. 
% Section \uppercase\expandafter{\romannumeral 6} briefly surveys related work.
% Section \uppercase\expandafter{\romannumeral 7} concludes the article.




