
\section{Conclusions}

In this paper, we presented a simple and novel localization technique based on half-plane intersection, HPI-SBL. 
The reference nodes sequence is computed by using TOA measurements of acoustic signals between the acoustic source and the reference nodes.
The half-planes are constructed by processing the node sequence, then turn the localization problem into half-plane intersection problem. 
Since our system runs on COTS smartphones and supports spontaneous setup, 
it has potential to enable a wide range of distributed acoustic localization system. 
 Besides the basic design,  robust HPI-SBL is proposed for further enhancing system robustness.
 Our system is verified and evaluated through analysis, extensive simulation as well as the test-bed experimentation.
 The test results have shown that the proposed method can effectively implement aoustic source localization with ad-hoc smartphone array.
 Our next step is to study the distributed localization method for ad-hoc smartphone array.
%Another future work is that further mining the information embedded in the node sequence to improve the robustness of localization system.

% \section{Acknowledge}

% This work is supported by Natural Science Foundation of China (Grants No. 61272524 and No.61202443) and the Fundamental Research Funds for the Central Universities (Grants No.DUT15QY05 and No.DUT15QY51). 
% This work is also supported by Specialized Research Fund for the Doctoral Program of Higher Education (Grant No. 20120041120049).

