

\section{DESIGN}

In this section, we firstly introduce the basic HPI-SBL method.
After the basic HPI-SBL method is proposed, we describe the probabilistic HPI-SBL method in the next subsection. Finally, weighted probabilistic HPI-SBL, the robust version of HPI-SBL is given.
%Finally, the computational complexity analysis of HPI-SBL is given.

\subsection{Basic HPI-SBL }

In this section, we introduce the basic sequence-based localization technique based on half-plane intersection.

Considering a sensor network in the 2D space with $N$ nodes, all nodes are $\bm{X} = \{ \bm{nod{e_1}}, \cdots ,\bm{nod{e_i}}, \cdots ,\bm{nod{e_N}}\}$, 
where any node $\bm{nod{e_i}}$ has its location coordinate denoted as $[{x_i},{y_i}]$. 
As showed in Fig. 2, an acoustic event occurs at $\bm{X_s}{\rm{ = [}}{x_s},{y_s}]$, ${d_i}$ is the distance from $\bm{nod{e_i}}$ to the acoustic source $\bm{S}$.
The node sequence is determined by the distances from nodes to the acoustic source $\bm{S}$.

Basic HPI-SBL turns the localization problem into half-plane intersection by processing the node sequence. 
The half-planes are constructed by the two adjacent nodes in the node sequences. 
We can find a solution to narrow the source region by making intersection of these half-planes. 
Given the following node sequence $NodeSeq( \cdots ,i,j,k,\cdots )$  obtained by time of arrival (TOA) information of the acoustic event,
we can get the distance sequence ${d_i} < {d_j} < {d_k} $ from  the acoustic source  to each node.
Just considering the adjacent node, we can have the following $N(N-1)/2$ linear constraints: $d_i<d_j$, $d_i<d_k$ and $d_j<d_k$, etc. 
We construct the corresponding $N(N-1)/2$  half-planes $H_{ij}$, $H_{ik}$ and $H_{jk}$, etc.
$d_i<d_j$ means that the acoustic source lies in the left half-plane $H_{ij}$ of the perpendicular bisector of $\bm{nod{e_i}}$ and $\bm{nod{e_j}}$.
The intersection of  the $N(N-1)/2$  half-planes is the final region of the acoustic source.
% \begin{algorithm}
% \caption{Basic HPI-SBL}
% \KwIn { The location of $N$ smartphones \\
% \hspace{0.41in} The node sequence of the acoustic source for\\
% \hspace{0.41in} $N$ reference nodes
% }
% \KwOut {Location of the acoustic source}

% \textbf{Step 1:} Setting the original region $R_S$ of acoustic source $S$: $R_S \leftarrow The whole space$;

% \textbf{Step 2:} Narrow the region by half-plane intersection: \\ 
% \hspace{0.41in} setting $\bm{A}$ and $\bm{b}$ by processing node sequence;\\
% \For{$i \leftarrow 1$ \textbf{to} $N-1$}
% {
% \For{$j \leftarrow i+1$ \textbf{to} $N$}
% {
% Use $node_i$ and $node_j$ to contruct half-plane $H_{ij}$;

% Solve half-plane intersection problem to get the target's location.

% $R_S$= $R_S$ U $H_{ij}$
% }
% }
% \textbf{Step 3:} the center of region $R_S$ is the target's location.
% \end{algorithm}

\subsection{Probabilistic HPI-SBL}

For the sake of presentation, until now we have described HPI-SBL in an ideal case where a complete and perfect node sequence can be obtained. 
In this section, we describe how to make HPI-SBL work well under more realistic conditions. 

%Basic HPI-SBL can find a solution to this feasibility program only if there is a region satisfying all the constraints. 
In the practical application, if two nodes are very close to each other along the direction of event propagation, they would detect the event almost simultaneously. 
In this case, the flip problem of node sequence may occur. 
%For instance, the true sequence is $NodeSeq ( \cdots ,i,j, \cdots )$, but the detected sequence is $NodeSeq ( \cdots ,j,i, \cdots )$.
Basic HPI-SBL might fail to find a feasible solution that satisfies all the constraints when sequence flip occurs. 
For example, as shown in Fig.2, the right middle region identifies by the node sequence $NodeSeq (D B C A)$. 
Once the order of node A and C is flipped in $NodeSeq (D B C A)$, the region corresponding to $NodeSeq (D B A C)$  does not exist, basic HPI-SBL can not give the accurate estimation.
In this section, we propose a robust solution to address the problem of sequence flip, called probabilistic HPI-SBL.

Considering the uncertainty of the measurement with the node sequence $NodeSeq( \cdots ,i,j, \cdots )$, the probability of ${d_i} < {d_j}$ can be described as  
 \begin{equation}\label{eqq1}
 p({d_i} < {d_j})>\alpha
 \end{equation}
Eq.\ref{eqq1} means that the probability of the acoustic source in the left side of the the half-plane $H_{ij}$ is $\alpha$, and the right side is $1-\alpha$. 
After dividing the localization space into some discrete grids, the weight of the grid point in the left side of the the half-plane $H_{ij}$ is $\alpha$, and the weight of the grid point in the right side of the the half-plane $H_{ij}$ is $1-\alpha$.

Besides these key constraints mentioned in basic HPI-SBL, some relaxed constraints are also provided some localization information.
For example, given $NodeSeq (A C B D)$, three key constraints $SA<SC$, $SC<SB$ and $SB<SD$ from the two adjacent nodes,
and three relaxed constraints $SA<SB$, $SA<SD$ and $SC<SD$ from the two nonadjacent nodes in the node sequence are achieved, respectively. 
Given the node sequences with $N$ nodes, all  $N(N - 1)/2$  constraints can be ultilized to further improve the localization robustness.
By processing  $N(N - 1)/2$ half-planes, we can compute the cumulative weight $w$ of each grid point $\bm{x}$

\begin{equation}\label{eq2}
w(\bm{x})=\sum\limits_{k=1} ^{N-1}\alpha(k).
\end{equation}
The region with the highest probability is the final region of the acoustic source
\begin{equation}\label{eq3}
\hat{\bm{X_s}}=arg \max_{\bm{x} \in R}w(\bm{x}).
\end{equation}
The centroid of the final region is the estimated location of the acoustic source.


%To summarize, the probabilistic HPI-SBL is presented in Algorithm 1. 
%The input are the node sequences and anchor locations; the output is the location of the acoustic source. 
%Step 1 sets the the initial weight of each discrete grid point. 
%Step 2 uses the node sequence to get multiple half-planes, then computes the cumulative weight.
%Step 3 adopts the center of the final intersection region as the location of acoustic source.

%\begin{algorithm}
%\caption{Probabilistic HPI-SBL}
%\KwIn { The location of $N$ smartphones \\
%\hspace{0.41in} The node sequence of the acoustic source for\\
%\hspace{0.41in} $N$ smartphones
%}
%\KwOut {Location of the acoustic source}

%\textbf{Step 1:} Setting the weight of each grid point  $w(k) \leftarrow 0$;

%\textbf{Step 2:} Computing the cumulative weight by processing node sequence: \\ 
%\hspace{0.0in} Contructing half planes by processing node sequence;\\
%\For{$i \leftarrow 1$ \textbf{to} $N-1$}
%{
%\For{$j \leftarrow i+1$ \textbf{to} $N$}
%{
%Calculating the probality of the line by the distance of $i,j$\\
%Contructing half plane $H_{ij}$\\
%Updating Weight $w(k)$ by the new probability for each grid point;\\
%}
%}
%\textbf{Step 3:} The center of region with the highest probability is the location of acoustic source.
%\end{algorithm}


\subsection{Weighted Probabilistic HPI-SBL}

 Considering the uncertainty of the measurement with the node sequence $NodeSeq( 1,\cdots ,i,j, \cdots,n )$, the probability of ${d_i} < {d_j}$ can be described as  
 \begin{equation}\label{eq1}
 p({d_i} < {d_j})>\alpha_{ij}
 \end{equation}

 In real condition, we consider the probabilities given by different half-plane exist differences. Here we call the line which divide the space as edge. We believe the edge constructed by $i,j$, if closer to the acoustic source, the half-plane divided by it is more essential to localization for determining a more accurate area, meawhile, is more likely to occur flip, we give $\alpha_{ij}$ a smaller value. As for the edges further to the source, its function is inferior in localization, but impossible to occur flip so more believable, we set $\alpha_{ij}$ a large value. In this way, utilizing the weighted probability to determine the acoustic source.
 
As the node sequence $NodeSeq( \cdots ,i,j, \cdots )$ is known to us, for every half-plane, the probability change dynamically according to the distance of the edge to the acoustic source. If the node $i,j$ that construct the half-plane in the sequence is 1 and $n$, as the relative distance is the largest, we set $\alpha_{ij}$ a largest value $\alpha$, if the relative distance of $i,j$ in the sequence is the smallest 1, we give $\alpha_{ij}$ a smallest probability $\beta$. As for other half-planes, we utilize the arithmetic propression between $\alpha$ and $\beta$ by the relative distance of $i,j$ to set their probabilities.

% \begin{algorithm}
% 	\caption{Weighted Probabilistic HPI-SBL}
% 	\KwIn { The location of $N$ smartphones \\
% 		\hspace{0.41in} The largest probability $\alpha$\\
% 		\hspace{0.41in} The smallest probability $\beta$\\
% 		\hspace{0.41in} The node sequence of the acoustic source for\\
% 		\hspace{0.41in} $N$ smartphones
% 	}
% 	\KwOut {Location of the acoustic source}
% 	
% 	\textbf{Step 1:} Setting the weight of each grid point  $w(k) \leftarrow 0$;
% 	
% 	\textbf{Step 2:} Giving half-plane probabilities and computing the cumulative weight by processing node sequence: \\ 
% 	\hspace{0.0in} Contructing half planes by processing node sequence;\\
% 	\For{$i \leftarrow 1$ \textbf{to} $N-1$}
% 	{
% 		\For{$j \leftarrow i+1$ \textbf{to} $N$}
% 		{
% 			Contructing half plane $H_{ij}$\\
% 			Calculating $\alpha_{ij}$ between $\alpha$ and $\beta$ according to the relative distance of $i,j$\\
% 			Updating Weight $w(k)$ for each grid point;\\
% 		}
% 	}
% 	\textbf{Step 3:} The center of region with the highest probability is the location of acoustic source.
% \end{algorithm}

\iffalse
 \subsection{Complexity Analysis}
 This section provides the complexity analysis for the proposed
 LPSBL design. It needs to be emphasized that
 LPSBL itself adopts an asymmetric design in which sensor
 nodes need only to detect and report the events. Therefore,
 we only analyze the computational cost on the node sequence
 processing side, where resources are plentiful.

 SBL: Calculating the Spearman’s coefficient and Kendall’s Tau
 between two sequences are $O(N)$ and $O(N^2)$ operations,
 respectively. Since the location sequence table is of size
 $O(N^4)$, searching through it takes $O(N^5)$ and $O(N^6)$ operations,
 respectively~\cite{yedavalli2008sequence}.

 RSBL: Discrete space grid  M  
\fi


